\documentclass{article}
\usepackage{amsmath}
\usepackage{listings}
\usepackage{xcolor}

% Define colors for code
\definecolor{lightgray}{rgb}{0.95,0.95,0.95}
\lstdefinestyle{mystyle}{
    backgroundcolor=\color{lightgray},
    commentstyle=\color{green},
    keywordstyle=\color{blue},
    numberstyle=\tiny\color{gray},
    stringstyle=\color{red},
    basicstyle=\ttfamily\footnotesize,
    breakatwhitespace=false,
    breaklines=true,
    numbers=left,
    numbersep=5pt,
    columns=fullflexible,
}
\lstset{style=mystyle}

\title{File Transfer System Using Sockets}
\author{Le Quang Huy - 22BI13190}
\date{November 29, 2024}

\begin{document}

\maketitle

\section{Introduction}
This document outlines the implementation of a file transfer system using Python sockets. The system consists of a server that sends a file to a client upon request.

\section{System Overview}
The system is composed of two main components:
\begin{itemize}
    \item A host that listens for incoming connections and sends a specified file.
    \item A client that connects to the server and receives the file.
\end{itemize}

\section{Server Implementation}
The server is implemented using the following code:

\begin{lstlisting}[language=Python, caption=Host Code]
import socket
import os

def handle_client(conn, address):
    print(f"Connected to {address}.")
    file_to_send = r'/media/icebear/Linux/DS/task1/transfer_file.txt'

    if os.path.exists(file_to_send):
        with open(file_to_send, 'rb') as file:
            print("Starting file transfer...")
            while (chunk := file.read(1024)):
                conn.send(chunk)
            print("File transfer complete.")
    else:
        print("File not found.")
        conn.send(b"File not found!")
    conn.close()

def host():
    server_host = '0.0.0.0'
    server_port = 5000
    server_socket = socket.socket(socket.AF_INET, socket.SOCK_STREAM)
    server_socket.setsockopt(socket.SOL_SOCKET, socket.SO_REUSEADDR, 1)
    server_socket.bind((server_host, server_port))
    server_socket.listen(1)
    print(f"Server listening on {server_host}:{server_port}...")

    while True:
        conn, address = server_socket.accept()
        handle_client(conn, address)

if __name__ == '__main__':
    host()
\end{lstlisting}

\section{Client Implementation}
The client is implemented as follows:

\begin{lstlisting}[language=Python, caption=Client Code]
import socket

def receive_file():
    server_host = '127.0.0.1'  
    server_port = 5000
    client_socket = socket.socket(socket.AF_INET, socket.SOCK_STREAM)

    client_socket.connect((server_host, server_port))

    received_filename = 'received_file.txt'
    with open(received_filename, 'wb') as file:
        print("Receiving file...")
        while True:
            data = client_socket.recv(1024)
            if not data:
                print("File transfer complete.")
                break
            file.write(data)

    print(f"File saved as {received_filename}.")
    client_socket.close()

if __name__ == '__main__':
    receive_file()
\end{lstlisting}

\section{Conclusion}
The code demonstrates a simple file transfer system using Python sockets, how to set up a host and a client for file communication. 

\end{document}