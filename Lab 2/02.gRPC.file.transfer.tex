\documentclass{article}
\usepackage{amsmath}
\usepackage{listings}
\usepackage{xcolor}

% Define colors for code
\definecolor{lightgray}{rgb}{0.95,0.95,0.95}
\lstdefinestyle{mystyle}{
    backgroundcolor=\color{lightgray},
    commentstyle=\color{green},
    keywordstyle=\color{blue},
    numberstyle=\tiny\color{gray},
    stringstyle=\color{red},
    basicstyle=\ttfamily\footnotesize,
    breakatwhitespace=false,
    breaklines=true,
    numbers=left,
    numbersep=5pt,
    columns=fullflexible,
}
\lstset{style=mystyle}

\title{File Transfer System Using gRPC}
\author{Le Quang Huy - 22BI13190}


\begin{document}

\maketitle

\section{Introduction}
This document outlines the implementation of a file transfer system using gRPC. The system consists of a server that sends a file to a client upon request.

\section{System Overview}
The system is composed of two main components:
\begin{itemize}
    \item A server that listens for incoming connections and sends a specified file.
    \item A client that connects to the server and receives the file.
\end{itemize}

\section{Server Implementation}
The server is implemented using the following code:

\begin{lstlisting}[language=Python, caption=Server Code]
import grpc
import file_transfer_pb2
import file_transfer_pb2_grpc
from concurrent import futures
import os

class FileTransferServicer(file_transfer_pb2_grpc.FileTransferServicer):
    def SendFile(self, request, context):
        file_to_send = r'D:\DS\task2\tranfer_file.txt'
        if os.path.exists(file_to_send):
            with open(file_to_send, 'rb') as file:
                return file_transfer_pb2.FileChunk(data=file.read(), filename='transfer_file.txt')
        else:
            context.set_details("File not found.")
            context.set_code(grpc.StatusCode.NOT_FOUND)
            return file_transfer_pb2.FileChunk()

def serve():
    server = grpc.server(futures.ThreadPoolExecutor(max_workers=10))
    file_transfer_pb2_grpc.add_FileTransferServicer_to_server(FileTransferServicer(), server)
    server.add_insecure_port('[::]:6000')
    print("Server listening on port 6000...")
    server.start()
    server.wait_for_termination()

if __name__ == '__main__':
    serve()
\end{lstlisting}

\section{Client Implementation}
The client is implemented as follows:

\begin{lstlisting}[language=Python, caption=Client Code]
import grpc
import file_transfer_pb2
import file_transfer_pb2_grpc

def receive_file():
    channel = grpc.insecure_channel('localhost:6000')
    stub = file_transfer_pb2_grpc.FileTransferStub(channel)
    response = stub.SendFile(file_transfer_pb2.Empty())

    if response.data:
        received_filename = 'received_file.txt'
        with open(received_filename, 'wb') as file:
            file.write(response.data)
        print(f"File saved as {received_filename}.")
    else:
        print("File not found on the server.")

if __name__ == '__main__':
    receive_file()
\end{lstlisting}

\section{Conclusion}
The code demonstrates a simple file transfer system using gRPC, showing how to set up a server and a client for file communication. 

\end{document}
